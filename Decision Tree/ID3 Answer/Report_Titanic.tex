\documentclass{article}
\linespread{1.25}

\usepackage[top = 1cm, right=2cm, left=2cm]{geometry}
\usepackage{sectsty}
\usepackage{amsmath}

\usepackage{graphicx}
\usepackage{xepersian}
\settextfont{B Nazanin}
\setlatinmonofont{CMU Serif}
%\setlatinmonofont{Times New Roman}
\setlatintextfont{Times New Roman}

% Set Latin Modern font for the bullets in itemizea
\newfontfamily\latinbullet{Latin Modern Roman}
\sectionfont{\fontsize{12}{15}\selectfont}
\DeclareMathOperator*{\argmax}{arg\,max}

\title{پاسخ تکلیف 
	\lr{Decision Tree (ID3)}
}
\author{درس یادگیری ماشین}
\date{
	امیرحسین ابوالحسنی\\
	400405003
	}
	
	
% Commands
\newcommand{\column}[1]{\lr{\textit{#1}}}
\renewcommand{\labelitemi}{{\latinbullet\textbullet}} % Use the bullet from Latin Modern font

\begin{document}
	\maketitle
	
	\begin{latin}
		\section{Suppose there is an attribute, "A," that consists of random values, and these
			values do not have any correlation with the class labels. Additionally, assume that
			"A" has a sufficient number of distinct values such that no two instances in the
			training dataset share the same value for "A." What would be the outcome if a
			decision tree is built using this attribute? What challenges or issues might arise in
			this scenario?}
	\end{latin}
	\noindent
	\subsection*{پاسخ}
	\subsubsection*{قسمت اول}
	با توجه به الگوریتم 
	\lr{ID3}،
	در ابتدا 
	\lr{information gain}
	ناشی از هر ویژگی را سنجیده و آن ویژگی که بیشترین 
	\lr{gain}
	را دارد انتخاب می‌کنیم\\
	(تعداد کلاس‌های هدف = \lr{k}):
	\[
	Gain(S, A) = Entropy(S) - \sum_{v \in A} \frac{| S_v |}{| S |}Entropy(S_v)
	\] 
	\[
	Entropy(S_v) = -\sum_{v} p_v\log (p_v) = P(S_v = 0) \log P(S_v = 0) + P(S_v = 1) \log P(S_v = 1) + \dots‌+ P(S_v = k) \log p(S_v = k)
	\]
	به علت یکتایی این ویژگی(کلید اصلی بودن) برای هر نمونه، همه ترم‌های $P(S_v = l) \log P(S_v = l)$ برابر با صفر می‌شود.  زیرا 
	\[
	P(S_v = l) = 0
	\]
	یا
	\[
	P(S_v = l) = 1
	\]
	 در نتیجه یکی از مضرب ها 0 خواهد شد و کل ترم‌ را 0 خواهد کرد. بدین صورت است که نتیجه می‌گیریم:
	\[
	Entropy(S_v) = 0
	\]
	و این ویژگی برای ریشه انتخاب می‌گردد:
	\[
	\argmax ~ \{Gain(S_v) |\forall A \in \text{\lr{Header}}\} = A
	\]
	در نتیجه این کار، ارتفاع درخت 1 شده و به تعداد مقادیر ویژگی 
	\lr{A}،
شاخه خواهیم داشت.
	
\end{document}